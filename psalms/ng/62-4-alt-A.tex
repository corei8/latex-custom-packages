2. Sitívit in te á\textit{ni}\textit{ma} \textbf{me}a, \ast\  quam multiplíciter ti\textit{bi} \textit{ca}\textit{ro} \textbf{me}a.\

3. In terra desérta, et ínvia, et inaquósa: \dag\  sic in sancto appá\textit{ru}\textit{i} \textbf{ti}bi, \ast\  ut vidérem virtútem tuam, et \textit{gló}\textit{ri}\textit{am} \textbf{tu}am.\

4. Quóniam mélior est misericórdia tua \textit{su}\textit{per} \textbf{vi}tas: \ast\  lábia \textit{me}\textit{a} \textit{lau}\textbf{dá}bunt te.\

5. Sic benedícam te in \textit{vi}\textit{ta} \textbf{me}a: \ast\  et in nómine tuo levá\textit{bo} \textit{ma}\textit{nus} \textbf{me}as.\

6. Sicut ádipe et pinguédine repleátur á\textit{ni}\textit{ma} \textbf{me}a: \ast\  et lábiis exsultatiónis lau\textit{dá}\textit{bit} \textit{os} \textbf{me}um.\

7. Si memor fui tui super stratum meum, \dag\  in matutínis medi\textit{tá}\textit{bor} \textbf{in} te: \ast\  quia fuísti \textit{ad}\textit{jú}\textit{tor} \textbf{me}us.\

8. Et in velaménto alárum tuárum exsultábo, \dag\  adhǽsit ánima \textit{me}\textit{a} \textbf{post} te: \ast\  me suscépit \textit{déx}\textit{te}\textit{ra} \textbf{tu}a.\

9. Ipsi vero in vanum quæsiérunt ánimam meam, \dag\  introíbunt in inferi\textit{ó}\textit{ra} \textbf{ter}ræ: \ast\  tradéntur in manus gládii, partes \textit{vúl}\textit{pi}\textit{um} \textbf{e}runt.\

10. Rex vero lætábitur in Deo, \dag\  laudabúntur omnes qui ju\textit{rant} \textit{in} \textbf{e}o: \ast\  quia obstrúctum est os loquén\textit{ti}\textit{um} \textit{in}\textbf{í}qua.\

