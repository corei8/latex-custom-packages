2. Quia ipse super mária fun\textit{dá}\textit{vit} \textbf{e}um: \ast\  et super flúmina præparávit \textbf{e}um.\

3. Quis ascéndet in \textit{mon}\textit{tem} \textbf{Dó}mini? \ast\  aut quis stabit in loco sancto \textbf{e}jus?\

4. Innocens mánibus et mundo corde, \dag\  qui non accépit in vano á\textit{ni}\textit{mam} \textbf{su}am, \ast\  nec jurávit in dolo próximo \textbf{su}o.\

5. Hic accípiet benedictió\textit{nem} \textit{a} \textbf{Dó}mino: \ast\  et misericórdiam a Deo, salutári \textbf{su}o.\

6. Hæc est generátio quærén\textit{ti}\textit{um} \textbf{e}um, \ast\  quæréntium fáciem Dei \textbf{Ja}cob.\

7. Attóllite portas, príncipes, vestras, \dag\  et elevámini, portæ \textit{æ}\textit{ter}\textbf{ná}les: \ast\  et introíbit Rex \textbf{gló}riæ.\

8. Quis est iste Rex glóriæ? \dag\  Dóminus for\textit{tis} \textit{et} \textbf{pot}ens: \ast\  Dóminus potens in \textbf{prǽ}lio.\

9. Attóllite portas, príncipes, vestras, \dag\  et elevámini, portæ \textit{æ}\textit{ter}\textbf{ná}les: \ast\  et introíbit Rex \textbf{gló}riæ.\

10. Quis est is\textit{te} \textit{Rex} \textbf{gló}riæ? \ast\  Dóminus virtútum ipse est Rex \textbf{gló}riæ.\

