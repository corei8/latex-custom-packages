2. Quia ipse super mária \textit{fun}\textit{dá}\textit{vit} \textbf{e}um: \ast\  et super flúmina præpará\textit{vit} \textbf{e}um.\

3. Quis ascéndet \textit{in} \textit{mon}\textit{tem} \textbf{Dó}mini? \ast\  aut quis stabit in loco sanc\textit{to} \textbf{e}jus?\

4. Innocens mánibus et mundo corde, \dag\  qui non accépit in vano \textit{á}\textit{ni}\textit{mam} \textbf{su}am, \ast\  nec jurávit in dolo próxi\textit{mo} \textbf{su}o.\

5. Hic accípiet benedicti\textit{ó}\textit{nem} \textit{a} \textbf{Dó}mino: \ast\  et misericórdiam a Deo, salutá\textit{ri} \textbf{su}o.\

6. Hæc est generátio quæ\textit{rén}\textit{ti}\textit{um} \textbf{e}um, \ast\  quæréntium fáciem De\textit{i} \textbf{Ja}cob.\

7. Attóllite portas, príncipes, vestras, \dag\  et elevámini, por\textit{tæ} \textit{æ}\textit{ter}\textbf{ná}les: \ast\  et introíbit \textit{Rex} \textbf{gló}riæ.\

8. Quis est iste Rex glóriæ? \dag\  Dóminus \textit{for}\textit{tis} \textit{et} \textbf{pot}ens: \ast\  Dóminus potens \textit{in} \textbf{prǽ}lio.\

9. Attóllite portas, príncipes, vestras, \dag\  et elevámini, por\textit{tæ} \textit{æ}\textit{ter}\textbf{ná}les: \ast\  et introíbit \textit{Rex} \textbf{gló}riæ.\

10. Quis est \textit{is}\textit{te} \textit{Rex} \textbf{gló}riæ? \ast\  Dóminus virtútum ipse est \textit{Rex} \textbf{gló}riæ.\

