2. Non avértas fáciem tuam \textbf{a} me: \ast\  in quacúmque die tríbulor, inclína ad me au\textit{rem} \textbf{tu}am.\

3. In quacúmque die invocáve\textbf{ro} te: \ast\  velóciter \textit{ex}\textbf{áu}\textbf{di} me.\

4. Quia defecérunt sicut fumus dies \textbf{me}i: \ast\  et ossa mea sicut crémium a\textit{ru}\textbf{é}runt.\

5. Percússus sum ut fœnum, et áruit cor \textbf{me}um: \ast\  quia oblítus sum comédere pa\textit{nem} \textbf{me}um.\

6. A voce gémitus \textbf{me}i: \ast\  adhǽsit os meum car\textit{ni} \textbf{me}æ.\

7. Símilis factus sum pellicáno soli\textbf{tú}dinis: \ast\  factus sum sicut nyctícorax in do\textit{mi}\textbf{cí}\textbf{li}o.\

8. Vigi\textbf{lá}vi, \ast\  et factus sum sicut passer solitárius \textit{in} \textbf{tec}to.\

9. Tota die exprobrábant mihi inimíci \textbf{me}i: \ast\  et qui laudábant me, advérsum me \textit{ju}\textbf{rá}bant.\

10. Quia cínerem tamquam panem mandu\textbf{cá}bam, \ast\  et potum meum cum fletu \textit{mi}\textbf{scé}bam.\

11. A fácie iræ et indignatiónis \textbf{tu}æ: \ast\  quia élevans al\textit{li}\textbf{sís}\textbf{ti} me.\

12. Dies mei sicut umbra declina\textbf{vé}runt: \ast\  et ego sicut fœ\textit{num} \textbf{á}\textbf{ru}i.\

13. Tu autem, Dómine, in ætérnum \textbf{pér}manes: \ast\  et memoriále tuum in generatiónem et genera\textit{ti}\textbf{ó}nem.\

14. Tu exsúrgens miseréberis \textbf{Si}on: \ast\  quia tempus miseréndi ejus, quia ve\textit{nit} \textbf{tem}pus.\

15. Quóniam placuérunt servis tuis lápides \textbf{e}jus: \ast\  et terræ ejus mise\textit{re}\textbf{bún}tur.\

16. Et timébunt gentes nomen tuum, \textbf{Dó}mine: \ast\  et omnes reges terræ glóri\textit{am} \textbf{tu}am.\

17. Quia ædificávit Dóminus \textbf{Si}on: \ast\  et vidébitur in glóri\textit{a} \textbf{su}a.\

18. Respéxit in oratiónem hu\textbf{mí}lium: \ast\  et non sprevit precem \textit{e}\textbf{ó}rum.\

19. Scribántur hæc in generatióne \textbf{ál}tera: \ast\  et pópulus qui creábitur, laudá\textit{bit} \textbf{Dó}\textbf{mi}num.\

20. Quia prospéxit de excélso sancto \textbf{su}o: \ast\  Dóminus de cælo in terram \textit{a}\textbf{spé}xit:\

21. Ut audíret gémitus compedi\textbf{tó}rum: \ast\  ut sólveret fílios inter\textit{emp}\textbf{tó}rum.\

22. Ut annúntient in Sion nomen \textbf{Dó}mini: \ast\  et laudem ejus in \textit{Je}\textbf{rú}\textbf{sa}lem.\

23. In conveniéndo pópulos in \textbf{u}num: \ast\  et reges ut sérvi\textit{ant} \textbf{Dó}\textbf{mi}no.\

24. Respóndit ei in via virtútis \textbf{su}æ: \ast\  Paucitátem diérum meórum núnti\textit{a} \textbf{mi}hi.\

25. Ne révoces me in dimídio diérum me\textbf{ó}rum: \ast\  in generatiónem et generatiónem an\textit{ni} \textbf{tu}i.\

26. Inítio tu, Dómine, terram fun\textbf{dás}ti: \ast\  et ópera mánuum tuárum \textit{sunt} \textbf{cæ}li.\

27. Ipsi períbunt, tu autem \textbf{pér}manes: \ast\  et omnes sicut vestiméntum ve\textit{te}\textbf{rá}scent.\

28. Et sicut opertórium mutábis eos, et muta\textbf{bún}tur: \ast\  tu autem idem ipse es, et anni tui non \textit{de}\textbf{fí}\textbf{ci}ent.\

29. Fílii servórum tuórum habi\textbf{tá}bunt: \ast\  et semen eórum in sǽculum di\textit{ri}\textbf{gé}tur.\

