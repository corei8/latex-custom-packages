2. Inténde voci oratiónis \textbf{me}æ: \ast\  Rex meus et \textbf{De}us \textbf{me}us.\

3. Quóniam ad te o\textbf{rá}bo: \ast\  Dómine, mane exáudies \textbf{vo}cem \textbf{me}am.\

4. Mane astábo tibi et vi\textbf{dé}bo: \ast\  quóniam non Deus volens iniqui\textbf{tá}tem \textbf{tu} es.\

5. Neque habitábit juxta te ma\textbf{lí}gnus: \ast\  neque permanébunt injústi ante \textbf{ó}culos \textbf{tu}os.\

6. Odísti omnes, qui operántur iniqui\textbf{tá}tem: \ast\  perdes omnes, qui lo\textbf{quún}tur men\textbf{dá}cium.\

7. Virum sánguinum et dolósum abominábitur \textbf{Dó}minus: \ast\  ego autem in multitúdine miseri\textbf{cór}diæ \textbf{tu}æ.\

8. Introíbo in domum \textbf{tu}am: \ast\  adorábo ad templum sanctum tuum in ti\textbf{mó}re \textbf{tu}o.\

9. Dómine, deduc me in justítia \textbf{tu}a: \ast\  propter inimícos meos dírige in conspéctu tuo \textbf{vi}am \textbf{me}am.\

10. Quóniam non est in ore eórum \textbf{vé}ritas: \ast\  cor e\textbf{ó}rum \textbf{va}num est.\

11. Sepúlcrum patens est guttur eórum, \dag\  linguis suis dolóse a\textbf{gé}bant, \ast\  júdica \textbf{il}los, \textbf{De}us.\

12. Décidant a cogitatiónibus suis, \dag\  secúndum multitúdinem impietátum eórum expélle \textbf{e}os, \ast\  quóniam irrita\textbf{vé}runt te, \textbf{Dó}mine.\

13. Et læténtur omnes, qui sperant \textbf{in} te, \ast\  in ætérnum exsultábunt: et habi\textbf{tá}bis in \textbf{e}is.\

14. Et gloriabúntur in te omnes, qui díligunt nomen \textbf{tu}um: \ast\  quóniam tu bene\textbf{dí}ces \textbf{jus}to.\

15. Dómine, ut scuto bonæ voluntátis \textbf{tu}æ \ast\  \textbf{co}ro\textbf{nás}ti nos.\

