2. Confessiónem, et decórem indu\textbf{ís}ti: \ast\  amíctus lúmine sicut \textbf{ves}ti\textbf{mén}to.\

3. Exténdens cælum sicut \textbf{pel}lem: \ast\  qui tegis aquis superi\textbf{ó}ra \textbf{e}jus.\

4. Qui ponis nubem ascénsum \textbf{tu}um: \ast\  qui ámbulas super \textbf{pen}nas ven\textbf{tó}rum.\

5. Qui facis ángelos tuos, \textbf{spí}ritus: \ast\  et minístros tuos \textbf{i}gnem u\textbf{rén}tem.\

6. Qui fundásti terram super stabilitátem \textbf{su}am: \ast\  non inclinábitur in \textbf{sǽ}culum \textbf{sǽ}culi.\

7. Abýssus, sicut vestiméntum, amíctus \textbf{e}jus: \ast\  super montes \textbf{sta}bunt \textbf{a}quæ.\

8. Ab increpatióne tua \textbf{fú}gient: \ast\  a voce tonítrui tui \textbf{for}mi\textbf{dá}bunt.\

9. Ascéndunt montes: et descéndunt \textbf{cam}pi \ast\  in locum, quem fun\textbf{dás}ti \textbf{e}is.\

10. Términum posuísti, quem non transgredi\textbf{én}tur: \ast\  neque converténtur ope\textbf{rí}re \textbf{ter}ram.\

11. Qui emíttis fontes in con\textbf{vál}libus: \ast\  inter médium móntium pertrans\textbf{í}bunt \textbf{a}quæ.\

12. Potábunt omnes béstiæ \textbf{a}gri: \ast\  exspectábunt ónagri in \textbf{si}ti \textbf{su}a.\

13. Super ea vólucres cæli habi\textbf{tá}bunt: \ast\  de médio petrárum \textbf{da}bunt \textbf{vo}ces.\

14. Rigans montes de superióribus \textbf{su}is: \ast\  de fructu óperum tuórum sati\textbf{á}bitur \textbf{ter}ra:\

15. Prodúcens fœnum ju\textbf{mén}tis: \ast\  et herbam servi\textbf{tú}ti \textbf{hó}minum:\

16. Ut edúcas panem de \textbf{ter}ra: \ast\  et vinum lætífi\textbf{cet} cor \textbf{hó}minis:\

17. Ut exhílaret fáciem in \textbf{ó}leo: \ast\  et panis cor hómi\textbf{nis} con\textbf{fír}met.\

18. Saturabúntur ligna campi, et cedri Líbani, quas plan\textbf{tá}vit: \ast\  illic pásseres ni\textbf{di}fi\textbf{cá}bunt.\

19. Heródii domus dux est e\textbf{ó}rum: \ast\  montes excélsi cervis: petra refúgium \textbf{he}ri\textbf{ná}ciis.\

20. Fecit lunam in \textbf{tém}pora: \ast\  sol cognóvit oc\textbf{cá}sum \textbf{su}um.\

21. Posuísti ténebras, et facta \textbf{est} nox: \ast\  in ipsa pertransíbunt omnes \textbf{bés}tiæ \textbf{sil}væ.\

22. Cátuli leónum rugiéntes, ut \textbf{rá}piant: \ast\  et quærant a Deo \textbf{es}cam \textbf{si}bi.\

23. Ortus est sol, et congre\textbf{gá}ti sunt: \ast\  et in cubílibus suis col\textbf{lo}ca\textbf{bún}tur.\

24. Exíbit homo ad opus \textbf{su}um: \ast\  et ad operatiónem suam \textbf{us}que ad \textbf{vés}perum.\

25. Quam magnificáta sunt ópera tua, \textbf{Dó}mine! \ast\  ómnia in sapiéntia fecísti: impléta est terra possessi\textbf{ó}ne \textbf{tu}a.\

26. Hoc mare magnum, et spatiósum \textbf{má}nibus: \ast\  illic reptília, quorum \textbf{non} est \textbf{nú}merus.\

27. Animália pusílla cum \textbf{ma}gnis: \ast\  illic naves \textbf{per}trans\textbf{í}bunt.\

28. Draco iste, quem formásti ad illudéndum \textbf{e}i: \ast\  ómnia a te exspéctant ut des illis \textbf{es}cam in \textbf{tém}pore.\

29. Dante te illis, \textbf{cól}ligent: \ast\  aperiénte te manum tuam, ómnia implebúntur \textbf{bo}ni\textbf{tá}te.\

30. Averténte autem te fáciem, turba\textbf{bún}tur: \ast\  áuferes spíritum eórum, et defícient, et in púlverem suum \textbf{re}ver\textbf{tén}tur.\

31. Emíttes spíritum tuum, et crea\textbf{bún}tur: \ast\  et renovábis \textbf{fá}ciem \textbf{ter}ræ.\

32. Sit glória Dómini in \textbf{sǽ}culum: \ast\  lætábitur Dóminus in o\textbf{pé}ribus \textbf{su}is:\

33. Qui réspicit terram, et facit eam \textbf{tré}mere: \ast\  qui tangit \textbf{mon}tes, et \textbf{fú}migant.\

34. Cantábo Dómino in vita \textbf{me}a: \ast\  psallam Deo meo, \textbf{quám}di\textbf{u} sum.\

35. Jucúndum sit ei elóquium \textbf{me}um: \ast\  ego vero delec\textbf{tá}bor in \textbf{Dó}mino.\

36. Defíciant peccatóres a terra, et iníqui ita ut \textbf{non} sint: \ast\  bénedic, ánima \textbf{me}a, \textbf{Dó}mino.\

