2. Confessiónem, et decórem \textbf{ind}u\textbf{ís}ti: \ast\  amíctus lúmine sicut \textit{ves}\textit{ti}\textbf{mén}to.\

3. Exténdens cælum \textbf{sic}ut \textbf{pel}lem: \ast\  qui tegis aquis superi\textit{ó}\textit{ra} \textbf{e}jus.\

4. Qui ponis nubem a\textbf{scén}sum \textbf{tu}um: \ast\  qui ámbulas super pen\textit{nas} \textit{ven}\textbf{tó}rum.\

5. Qui facis ángelos \textbf{tu}os, \textbf{spí}ritus: \ast\  et minístros tuos i\textit{gnem} \textit{u}\textbf{rén}tem.\

6. Qui fundásti terram super stabili\textbf{tá}tem \textbf{su}am: \ast\  non inclinábitur in sǽ\textit{cu}\textit{lum} \textbf{sǽ}culi.\

7. Abýssus, sicut vestiméntum, a\textbf{míc}tus \textbf{e}jus: \ast\  super montes \textit{sta}\textit{bunt} \textbf{a}quæ.\

8. Ab increpatióne \textbf{tu}a \textbf{fú}gient: \ast\  a voce tonítrui tui \textit{for}\textit{mi}\textbf{dá}bunt.\

9. Ascéndunt montes: et de\textbf{scén}dunt \textbf{cam}pi \ast\  in locum, quem fun\textit{dás}\textit{ti} \textbf{e}is.\

10. Términum posuísti, quem non trans\textbf{gre}di\textbf{én}tur: \ast\  neque converténtur ope\textit{rí}\textit{re} \textbf{ter}ram.\

11. Qui emíttis fontes \textbf{in} con\textbf{vál}libus: \ast\  inter médium móntium pertrans\textit{í}\textit{bunt} \textbf{a}quæ.\

12. Potábunt omnes \textbf{bés}tiæ \textbf{a}gri: \ast\  exspectábunt ónagri in \textit{si}\textit{ti} \textbf{su}a.\

13. Super ea vólucres cæli \textbf{ha}bi\textbf{tá}bunt: \ast\  de médio petrárum \textit{da}\textit{bunt} \textbf{vo}ces.\

14. Rigans montes de superi\textbf{ó}ribus \textbf{su}is: \ast\  de fructu óperum tuórum satiá\textit{bi}\textit{tur} \textbf{ter}ra:\

15. Prodúcens \textbf{fœ}num ju\textbf{mén}tis: \ast\  et herbam servi\textit{tú}\textit{ti} \textbf{hó}minum:\

16. Ut edúcas \textbf{pa}nem de \textbf{ter}ra: \ast\  et vinum lætífi\textit{cet} \textit{cor} \textbf{hó}minis:\

17. Ut exhílaret fáci\textbf{em} in \textbf{ó}leo: \ast\  et panis cor hómi\textit{nis} \textit{con}\textbf{fír}met.\

18. Saturabúntur ligna campi, et cedri Líbani, \textbf{quas} plan\textbf{tá}vit: \ast\  illic pásseres ni\textit{di}\textit{fi}\textbf{cá}bunt.\

19. Heródii domus dux \textbf{est} e\textbf{ó}rum: \ast\  montes excélsi cervis: petra refúgium \textit{he}\textit{ri}\textbf{ná}ciis.\

20. Fecit \textbf{lu}nam in \textbf{tém}pora: \ast\  sol cognóvit oc\textit{cá}\textit{sum} \textbf{su}um.\

21. Posuísti ténebras, et \textbf{fac}ta \textbf{est} nox: \ast\  in ipsa pertransíbunt omnes bés\textit{ti}\textit{æ} \textbf{sil}væ.\

22. Cátuli leónum rugi\textbf{én}tes, ut \textbf{rá}piant: \ast\  et quærant a Deo \textit{es}\textit{cam} \textbf{si}bi.\

23. Ortus est sol, et \textbf{con}gre\textbf{gá}ti sunt: \ast\  et in cubílibus suis col\textit{lo}\textit{ca}\textbf{bún}tur.\

24. Exíbit homo ad \textbf{o}pus \textbf{su}um: \ast\  et ad operatiónem suam us\textit{que} \textit{ad} \textbf{vés}perum.\

25. Quam magnificáta sunt ópera \textbf{tu}a, \textbf{Dó}mine! \ast\  ómnia in sapiéntia fecísti: impléta est terra possessi\textit{ó}\textit{ne} \textbf{tu}a.\

26. Hoc mare magnum, et spati\textbf{ó}sum \textbf{má}nibus: \ast\  illic reptília, quorum \textit{non} \textit{est} \textbf{nú}merus.\

27. Animália pu\textbf{síl}la cum \textbf{ma}gnis: \ast\  illic naves \textit{per}\textit{trans}\textbf{í}bunt.\

28. Draco iste, quem formásti ad illu\textbf{dén}dum \textbf{e}i: \ast\  ómnia a te exspéctant ut des illis es\textit{cam} \textit{in} \textbf{tém}pore.\

29. Dante te \textbf{il}lis, \textbf{cól}ligent: \ast\  aperiénte te manum tuam, ómnia implebúntur \textit{bo}\textit{ni}\textbf{tá}te.\

30. Averténte autem te fáciem, \textbf{tur}ba\textbf{bún}tur: \ast\  áuferes spíritum eórum, et defícient, et in púlverem suum \textit{re}\textit{ver}\textbf{tén}tur.\

31. Emíttes spíritum tuum, et \textbf{cre}a\textbf{bún}tur: \ast\  et renovábis fá\textit{ci}\textit{em} \textbf{ter}ræ.\

32. Sit glória Dómi\textbf{ni} in \textbf{sǽ}culum: \ast\  lætábitur Dóminus in opé\textit{ri}\textit{bus} \textbf{su}is:\

33. Qui réspicit terram, et facit \textbf{e}am \textbf{tré}mere: \ast\  qui tangit mon\textit{tes}, \textit{et} \textbf{fú}migant.\

34. Cantábo Dómino in \textbf{vi}ta \textbf{me}a: \ast\  psallam Deo meo, \textit{quám}\textit{di}\textbf{u} sum.\

35. Jucúndum sit ei e\textbf{ló}quium \textbf{me}um: \ast\  ego vero delectá\textit{bor} \textit{in} \textbf{Dó}mino.\

36. Defíciant peccatóres a terra, et iníqui \textbf{i}ta ut \textbf{non} sint: \ast\  bénedic, ánima \textit{me}\textit{a}, \textbf{Dó}mino.\

