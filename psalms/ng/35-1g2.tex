2. Quóniam dolóse egit in con\textbf{spéc}tu \textbf{e}jus: \ast\  ut inveniátur iníquitas e\textit{jus} \textit{ad} \textbf{ó}dium.\

3. Verba oris ejus iníqui\textbf{tas}, et \textbf{do}lus: \ast\  nóluit intellígere ut \textit{be}\textit{ne} \textbf{á}geret.\

4. Iniquitátem meditátus est in cu\textbf{bí}li \textbf{su}o: \ast\  ástitit omni viæ non bonæ, malítiam autem \textit{non} \textit{o}\textbf{dí}vit.\

5. Dómine, in cælo miseri\textbf{cór}dia \textbf{tu}a: \ast\  et véritas tua us\textit{que} \textit{ad} \textbf{nu}bes.\

6. Justítia tua sicut \textbf{mon}tes \textbf{De}i: \ast\  judícia tua a\textit{býs}\textit{sus} \textbf{mul}ta.\

7. Hómines, et juménta sal\textbf{vá}bis, \textbf{Dó}mine: \ast\  quemádmodum multiplicásti misericórdiam \textit{tu}\textit{am}, \textbf{De}us,\

8. Fílii \textbf{au}tem \textbf{hó}minum, \ast\  in tégmine alárum tuá\textit{rum} \textit{spe}\textbf{rá}bunt.\

9. Inebriabúntur ab ubertáte \textbf{do}mus \textbf{tu}æ: \ast\  et torrénte voluptátis tuæ po\textit{tá}\textit{bis} \textbf{e}os.\

10. Quóniam apud te \textbf{est} fons \textbf{vi}tæ: \ast\  et in lúmine tuo vidé\textit{bi}\textit{mus} \textbf{lu}men.\

11. Præténde misericórdiam tuam sci\textbf{én}ti\textbf{bus} te, \ast\  et justítiam tuam his, qui rec\textit{to} \textit{sunt} \textbf{cor}de.\

12. Non véniat mihi \textbf{pes} su\textbf{pér}biæ: \ast\  et manus peccatóris non \textit{mó}\textit{ve}\textbf{at} me.\

13. Ibi cecidérunt qui operántur in\textbf{i}qui\textbf{tá}tem: \ast\  expúlsi sunt, nec potu\textit{é}\textit{runt} \textbf{sta}re.\

