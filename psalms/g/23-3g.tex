2. Quia ipse super mária fun\textbf{dá}vit \textbf{e}um: \ast\  et super flúmina præpa\textit{rá}\textit{vit} \textbf{e}um.\

3. Quis ascéndet in \textbf{mon}tem \textbf{Dó}\textbf{mi}ni? \ast\  aut quis stabit in loco \textit{sanc}\textit{to} \textbf{e}jus?\

4. Innocens mánibus et mundo corde, \dag\  qui non accépit in vano \textbf{á}nimam \textbf{su}am, \ast\  nec jurávit in dolo pró\textit{xi}\textit{mo} \textbf{su}o.\

5. Hic accípiet benedicti\textbf{ó}nem a \textbf{Dó}\textbf{mi}no: \ast\  et misericórdiam a Deo, salu\textit{tá}\textit{ri} \textbf{su}o.\

6. Hæc est generátio quæ\textbf{rén}tium \textbf{e}um, \ast\  quæréntium fáciem \textit{De}\textit{i} \textbf{Ja}cob.\

7. Attóllite portas, príncipes, vestras, \dag\  et elevámini, portæ \textbf{æ}ter\textbf{ná}les: \ast\  et introí\textit{bit} \textit{Rex} \textbf{gló}riæ.\

8. Quis est iste Rex glóriæ? \dag\  Dóminus \textbf{for}tis et \textbf{pot}ens: \ast\  Dóminus pot\textit{ens} \textit{in} \textbf{prǽ}lio.\

9. Attóllite portas, príncipes, vestras, \dag\  et elevámini, portæ \textbf{æ}ter\textbf{ná}les: \ast\  et introí\textit{bit} \textit{Rex} \textbf{gló}riæ.\

10. Quis est \textbf{is}te Rex \textbf{gló}\textbf{ri}æ? \ast\  Dóminus virtútum ipse \textit{est} \textit{Rex} \textbf{gló}riæ.\

11. Glória \textbf{Pa}tri, et \textbf{Fí}\textbf{li}o, \ast\  et Spirí\textit{tu}\textit{i} \textbf{Sanc}to.\

12. Sicut erat in princípio, et \textbf{nunc}, et \textbf{sem}per, \ast\  et in sǽcula sæcu\textit{ló}\textit{rum}. \textbf{A}men.\

