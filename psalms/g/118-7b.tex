2. Beáti, qui scrutántur testi\textbf{mó}nia \textbf{e}jus: \ast\  in toto corde ex\textbf{quí}runt \textbf{e}um.\

3. Non enim qui operántur in\textbf{i}qui\textbf{tá}tem, \ast\  in viis ejus am\textbf{bu}la\textbf{vé}runt.\

4. \textbf{Tu} man\textbf{dás}ti \ast\  mandáta tua custo\textbf{dí}ri \textbf{ni}mis.\

5. Utinam dirigántur \textbf{vi}æ \textbf{me}æ, \ast\  ad custodiéndas justificati\textbf{ó}nes \textbf{tu}as!\

6. Tunc \textbf{non} con\textbf{fún}dar, \ast\  cum perspéxero in ómnibus man\textbf{dá}tis \textbf{tu}is.\

7. Confitébor tibi in directi\textbf{ó}ne \textbf{cor}dis: \ast\  in eo quod dídici judícia jus\textbf{tí}tiæ \textbf{tu}æ.\

8. Justificatiónes \textbf{tu}as cus\textbf{tó}diam: \ast\  non me derelínquas \textbf{us}que\textbf{quá}que.\

9. In quo córrigit adolescéntior \textbf{vi}am \textbf{su}am? \ast\  in custodiéndo ser\textbf{mó}nes \textbf{tu}os.\

10. In toto corde meo \textbf{ex}qui\textbf{sí}vi te: \ast\  ne repéllas me a man\textbf{dá}tis \textbf{tu}is.\

11. In corde meo abscóndi e\textbf{ló}quia \textbf{tu}a: \ast\  ut non \textbf{pec}cem \textbf{ti}bi.\

12. Bene\textbf{díc}tus es, \textbf{Dó}mine: \ast\  doce me justificati\textbf{ó}nes \textbf{tu}as.\

13. In \textbf{lá}biis \textbf{me}is, \ast\  pronuntiávi ómnia judícia \textbf{o}ris \textbf{tu}i.\

14. In via testimoniórum tuórum \textbf{de}lec\textbf{tá}tus sum, \ast\  sicut in ómni\textbf{bus} di\textbf{ví}tiis.\

15. In mandátis tuis \textbf{ex}er\textbf{cé}bor: \ast\  et considerábo \textbf{vi}as \textbf{tu}as.\

16. In justificatiónibus tuis \textbf{me}di\textbf{tá}bor: \ast\  non oblivíscar ser\textbf{mó}nes \textbf{tu}os.\

17. Retríbue servo tuo, vi\textbf{ví}fi\textbf{ca} me: \ast\  et custódiam ser\textbf{mó}nes \textbf{tu}os.\

18. Revéla \textbf{ó}culos \textbf{me}os: \ast\  et considerábo mirabília de \textbf{le}ge \textbf{tu}a.\

19. Incola ego \textbf{sum} in \textbf{ter}ra: \ast\  non abscóndas a me man\textbf{dá}ta \textbf{tu}a.\

20. Concupívit ánima mea desideráre justificati\textbf{ó}nes \textbf{tu}as, \ast\  in \textbf{om}ni \textbf{tém}pore.\

21. Incre\textbf{pás}ti su\textbf{pér}bos: \ast\  maledícti qui declínant a man\textbf{dá}tis \textbf{tu}is.\

22. Aufer a me oppróbrium, \textbf{et} con\textbf{témp}tum: \ast\  quia testimónia tua \textbf{ex}qui\textbf{sí}vi.\

23. Etenim sedérunt príncipes, et advérsum me \textbf{lo}que\textbf{bán}tur: \ast\  servus autem tuus exercebátur in justificati\textbf{ó}nibus \textbf{tu}is.\

24. Nam et testimónia tua medi\textbf{tá}tio \textbf{me}a est: \ast\  et consílium meum justificati\textbf{ó}nes \textbf{tu}æ.\

25. Adhǽsit paviménto \textbf{á}nima \textbf{me}a: \ast\  vivífica me secúndum \textbf{ver}bum \textbf{tu}um.\

26. Vias meas enuntiávi et \textbf{ex}au\textbf{dís}ti me: \ast\  doce me justificati\textbf{ó}nes \textbf{tu}as.\

27. Viam justificatiónum tuárum \textbf{ín}stru\textbf{e} me: \ast\  et exercébor in mira\textbf{bí}libus \textbf{tu}is.\

28. Dormitávit ánima \textbf{me}a præ \textbf{tǽ}dio: \ast\  confírma me in \textbf{ver}bis \textbf{tu}is.\

29. Viam iniquitátis \textbf{á}move \textbf{a} me: \ast\  et de lege tua mise\textbf{ré}re \textbf{me}i.\

30. Viam veri\textbf{tá}tis e\textbf{lé}gi: \ast\  judícia tua non \textbf{sum} ob\textbf{lí}tus.\

31. Adhǽsi testimóniis \textbf{tu}is \textbf{Dó}mine: \ast\  noli \textbf{me} con\textbf{fún}dere.\

32. Viam mandatórum tu\textbf{ó}rum cu\textbf{cúr}ri: \ast\  cum dila\textbf{tás}ti cor \textbf{me}um.\

33. Legem pone mihi, Dómine, viam justificati\textbf{ó}num tu\textbf{á}rum: \ast\  et exquíram \textbf{e}am \textbf{sem}per.\

34. Da mihi intelléctum, et scrutábor \textbf{le}gem \textbf{tu}am: \ast\  et custódiam illam in toto \textbf{cor}de \textbf{me}o.\

35. Deduc me in sémitam manda\textbf{tó}rum tu\textbf{ó}rum: \ast\  quia \textbf{ip}sam \textbf{vó}lui.\

36. Inclína cor meum in testi\textbf{mó}nia \textbf{tu}a: \ast\  et non in \textbf{a}va\textbf{rí}tiam.\

37. Avérte óculos meos ne vídeant \textbf{va}ni\textbf{tá}tem: \ast\  in via tua vi\textbf{ví}fi\textbf{ca} me.\

38. Státue servo tuo e\textbf{ló}quium \textbf{tu}um, \ast\  in ti\textbf{mó}re \textbf{tu}o.\

39. Amputa oppróbrium meum quod \textbf{su}spi\textbf{cá}tus sum: \ast\  quia judícia \textbf{tu}a ju\textbf{cún}da.\

40. Ecce concupívi man\textbf{dá}ta \textbf{tu}a: \ast\  in æquitáte tua vi\textbf{ví}fi\textbf{ca} me.\

41. Et véniat super me misericórdia \textbf{tu}a, \textbf{Dó}mine: \ast\  salutáre tuum secúndum e\textbf{ló}quium \textbf{tu}um.\

42. Et respondébo exprobrántibus \textbf{mi}hi \textbf{ver}bum: \ast\  quia sperávi in ser\textbf{mó}nibus \textbf{tu}is.\

43. Et ne áuferas de ore meo verbum veritátis \textbf{us}que\textbf{quá}que: \ast\  quia in judíciis tuis su\textbf{per}spe\textbf{rá}vi.\

44. Et custódiam legem \textbf{tu}am \textbf{sem}per: \ast\  in sǽculum et in \textbf{sǽ}culum \textbf{sǽ}culi.\

45. Et ambulábam in \textbf{la}ti\textbf{tú}dine: \ast\  quia mandáta tua \textbf{ex}qui\textbf{sí}vi.\

46. Et loquébar in testimóniis tuis in con\textbf{spéc}tu \textbf{re}gum: \ast\  et non \textbf{con}fun\textbf{dé}bar.\

47. Et meditábar in man\textbf{dá}tis \textbf{tu}is, \ast\  \textbf{quæ} di\textbf{lé}xi.\

48. Et levávi manus meas ad mandáta tua, \textbf{quæ} di\textbf{lé}xi: \ast\  et exercébar in justificati\textbf{ó}nibus \textbf{tu}is.\

49. Memor esto verbi tui \textbf{ser}vo \textbf{tu}o, \ast\  in quo mihi \textbf{spem} de\textbf{dís}ti.\

50. Hæc me consoláta est in humili\textbf{tá}te \textbf{me}a: \ast\  quia elóquium tuum vi\textbf{vi}fi\textbf{cá}vit me.\

51. Supérbi iníque agébant \textbf{us}que\textbf{quá}que: \ast\  a lege autem tua non \textbf{de}cli\textbf{ná}vi.\

52. Memor fui judiciórum tuórum a \textbf{sǽ}culo, \textbf{Dó}mine: \ast\  et \textbf{con}so\textbf{lá}tus sum.\

53. Deféctio \textbf{té}nu\textbf{it} me, \ast\  pro peccatóribus derelinquéntibus \textbf{le}gem \textbf{tu}am.\

54. Cantábiles mihi erant justificati\textbf{ó}nes \textbf{tu}æ, \ast\  in loco peregrinati\textbf{ó}nis \textbf{me}æ.\

55. Memor fui nocte nóminis \textbf{tu}i, \textbf{Dó}mine: \ast\  et custodívi \textbf{le}gem \textbf{tu}am.\

56. Hæc \textbf{fac}ta est \textbf{mi}hi: \ast\  quia justificatiónes tuas \textbf{ex}qui\textbf{sí}vi.\

57. Pórtio \textbf{me}a, \textbf{Dó}mine, \ast\  dixi custodíre \textbf{le}gem \textbf{tu}am.\

58. Deprecátus sum fáciem tuam in toto \textbf{cor}de \textbf{me}o: \ast\  miserére mei secúndum e\textbf{ló}quium \textbf{tu}um.\

59. Cogitávi \textbf{vi}as \textbf{me}as: \ast\  et convérti pedes meos in testi\textbf{mó}nia \textbf{tu}a.\

60. Parátus sum, et non \textbf{sum} tur\textbf{bá}tus: \ast\  ut custódiam man\textbf{dá}ta \textbf{tu}a.\

61. Funes peccatórum circum\textbf{plé}xi \textbf{sunt} me: \ast\  et legem tuam non \textbf{sum} ob\textbf{lí}tus.\

62. Média nocte surgébam ad confi\textbf{tén}dum \textbf{ti}bi: \ast\  super judícia justificati\textbf{ó}nis \textbf{tu}æ.\

63. Párticeps ego sum ómnium ti\textbf{mén}ti\textbf{um} te: \ast\  et custodiéntium man\textbf{dá}ta \textbf{tu}a.\

64. Misericórdia tua, Dómine, \textbf{ple}na est \textbf{ter}ra: \ast\  justificatiónes \textbf{tu}as \textbf{do}ce me.\

65. Bonitátem fecísti cum servo \textbf{tu}o, \textbf{Dó}mine: \ast\  secúndum \textbf{ver}bum \textbf{tu}um.\

66. Bonitátem et disciplínam et sci\textbf{én}tiam \textbf{do}ce me: \ast\  quia mandátis \textbf{tu}is \textbf{cré}didi.\

67. Priúsquam humiliárer \textbf{e}go de\textbf{lí}qui: \ast\  proptérea elóquium tuum \textbf{cus}to\textbf{dí}vi.\

68. \textbf{Bo}nus \textbf{es} tu: \ast\  et in bonitáte tua doce me justificati\textbf{ó}nes \textbf{tu}as.\

69. Multiplicáta est super me iníquitas \textbf{su}per\textbf{bó}rum: \ast\  ego autem in toto corde meo scrutábor man\textbf{dá}ta \textbf{tu}a.\

70. Coagulátum est sicut lac \textbf{cor} e\textbf{ó}rum: \ast\  ego vero legem tuam \textbf{me}di\textbf{tá}tus sum.\

71. Bonum mihi quia hu\textbf{mi}li\textbf{ás}ti me: \ast\  ut discam justificati\textbf{ó}nes \textbf{tu}as.\

72. Bonum mihi lex \textbf{o}ris \textbf{tu}i: \ast\  super míllia auri \textbf{et} ar\textbf{gén}ti.\

73. Manus tuæ fecérunt me, et \textbf{plas}ma\textbf{vé}runt me: \ast\  da mihi intelléctum, et discam man\textbf{dá}ta \textbf{tu}a.\

74. Qui timent te vidébunt me et \textbf{læ}ta\textbf{bún}tur: \ast\  quia in verba tua su\textbf{per}spe\textbf{rá}vi.\

75. Cognóvi, Dómine, quia ǽquitas ju\textbf{dí}cia \textbf{tu}a: \ast\  et in veritáte tua hu\textbf{mi}li\textbf{ás}ti me.\

76. Fiat misericórdia tua ut \textbf{con}so\textbf{lé}tur me: \ast\  secúndum elóquium tuum \textbf{ser}vo \textbf{tu}o.\

77. Véniant mihi miseratiónes \textbf{tu}æ, et \textbf{vi}vam: \ast\  quia lex tua medi\textbf{tá}tio \textbf{me}a est.\

78. Confundántur supérbi, quia injúste iniquitátem fe\textbf{cé}runt \textbf{in} me: \ast\  ego autem exercébor in man\textbf{dá}tis \textbf{tu}is.\

79. Convertántur \textbf{mi}hi ti\textbf{mén}tes te: \ast\  et qui novérunt testi\textbf{mó}nia \textbf{tu}a.\

80. Fiat cor meum immaculátum in justificati\textbf{ó}nibus \textbf{tu}is, \ast\  ut \textbf{non} con\textbf{fún}dar.\

81. Defécit in salutáre tuum \textbf{á}nima \textbf{me}a: \ast\  et in verbum tuum su\textbf{per}spe\textbf{rá}vi.\

82. Defecérunt óculi mei in e\textbf{ló}quium \textbf{tu}um: \ast\  dicéntes: Quando conso\textbf{lá}be\textbf{ris} me?\

83. Quia factus sum sicut uter \textbf{in} pru\textbf{í}na: \ast\  justificatiónes tuas non \textbf{sum} ob\textbf{lí}tus.\

84. Quot sunt dies \textbf{ser}vi \textbf{tu}i? \ast\  quando fácies de persequéntibus \textbf{me} ju\textbf{dí}cium?\

85. Narravérunt mihi iníqui fabu\textbf{la}ti\textbf{ó}nes: \ast\  sed non \textbf{ut} lex \textbf{tu}a.\

86. Omnia mandáta \textbf{tu}a \textbf{vé}ritas: \ast\  iníque persecúti sunt me, \textbf{ád}ju\textbf{va} me.\

87. Paulo minus consummavérunt \textbf{me} in \textbf{ter}ra: \ast\  ego autem non derelíqui man\textbf{dá}ta \textbf{tu}a.\

88. Secúndum misericórdiam tuam vi\textbf{ví}fi\textbf{ca} me: \ast\  et custódiam testimónia \textbf{o}ris \textbf{tu}i.\

89. In æ\textbf{tér}num, \textbf{Dó}mine, \ast\  verbum tuum pérma\textbf{net} in \textbf{cæ}lo.\

90. In generatiónem et generatiónem \textbf{vé}ritas \textbf{tu}a: \ast\  fundásti \textbf{ter}ram, et \textbf{pér}manet.\

91. Ordinatióne tua perse\textbf{vé}rat \textbf{di}es: \ast\  quóniam ómnia \textbf{sér}viunt \textbf{ti}bi.\

92. Nisi quod lex tua medi\textbf{tá}tio \textbf{me}a est: \ast\  tunc forte periíssem in humili\textbf{tá}te \textbf{me}a.\

93. In ætérnum non oblivíscar justificati\textbf{ó}nes \textbf{tu}as: \ast\  quia in ipsis vi\textbf{vi}fi\textbf{cás}ti me.\

94. Tuus sum ego, \textbf{sal}vum \textbf{me} fac: \ast\  quóniam justificatiónes tuas \textbf{ex}qui\textbf{sí}vi.\

95. Me exspectavérunt peccatóres ut \textbf{pér}de\textbf{rent} me: \ast\  testimónia tua \textbf{in}tel\textbf{lé}xi.\

96. Omnis consummatiónis \textbf{vi}di \textbf{fi}nem: \ast\  latum mandátum \textbf{tu}um \textbf{ni}mis.\

97. Quómodo diléxi legem \textbf{tu}am, \textbf{Dó}mine? \ast\  tota die medi\textbf{tá}tio \textbf{me}a est.\

98. Super inimícos meos prudéntem me fecísti man\textbf{dá}to \textbf{tu}o: \ast\  quia in æ\textbf{tér}num \textbf{mi}hi est.\

99. Super omnes docéntes me \textbf{in}tel\textbf{lé}xi: \ast\  quia testimónia tua medi\textbf{tá}tio \textbf{me}a est.\

100. Super senes \textbf{in}tel\textbf{lé}xi: \ast\  quia mandáta \textbf{tu}a quæ\textbf{sí}vi.\

101. Ab omni via mala prohíbui \textbf{pe}des \textbf{me}os: \ast\  ut custódiam \textbf{ver}ba \textbf{tu}a.\

102. A judíciis tuis non \textbf{de}cli\textbf{ná}vi: \ast\  quia tu legem posu\textbf{ís}ti \textbf{mi}hi.\

103. Quam dúlcia fáucibus meis e\textbf{ló}quia \textbf{tu}a, \ast\  super mel \textbf{o}ri \textbf{me}o!\

104. A mandátis tuis \textbf{in}tel\textbf{lé}xi: \ast\  proptérea odívi omnem viam in\textbf{i}qui\textbf{tá}tis.\

105. Lucérna pédibus meis \textbf{ver}bum \textbf{tu}um, \ast\  et lumen \textbf{sé}mitis \textbf{me}is.\

106. Ju\textbf{rá}vi, et \textbf{stá}tui \ast\  custodíre judícia jus\textbf{tí}tiæ \textbf{tu}æ.\

107. Humiliátus sum usque\textbf{quá}que, \textbf{Dó}mine: \ast\  vivífica me secúndum \textbf{ver}bum \textbf{tu}um.\

108. Voluntária oris mei benepláci\textbf{ta} fac, \textbf{Dó}mine: \ast\  et judícia \textbf{tu}a \textbf{do}ce me.\

109. Anima mea in mánibus \textbf{me}is \textbf{sem}per: \ast\  et legem tuam non \textbf{sum} ob\textbf{lí}tus.\

110. Posuérunt peccatóres \textbf{lá}queum \textbf{mi}hi: \ast\  et de mandátis tuis \textbf{non} er\textbf{rá}vi.\

111. Hereditáte acquisívi testimónia tua \textbf{in} æ\textbf{tér}num: \ast\  quia exsultátio \textbf{cor}dis \textbf{me}i sunt.\

112. Inclinávi cor meum ad faciéndas justificatiónes tuas \textbf{in} æ\textbf{tér}num, \ast\  propter retri\textbf{bu}ti\textbf{ó}nem.\

113. Iníquos \textbf{ó}dio \textbf{há}bui: \ast\  et legem \textbf{tu}am di\textbf{lé}xi.\

114. Adjútor et suscéptor \textbf{me}us \textbf{es} tu: \ast\  et in verbum tuum su\textbf{per}spe\textbf{rá}vi.\

115. Declináte a \textbf{me}, ma\textbf{lí}gni: \ast\  et scrutábor mandáta \textbf{De}i \textbf{me}i.\

116. Súscipe me secúndum elóquium \textbf{tu}um, et \textbf{vi}vam: \ast\  et non confúndas me ab exspectati\textbf{ó}ne \textbf{me}a.\

117. Adjuva me, et \textbf{sal}vus \textbf{e}ro: \ast\  et meditábor in justificatiónibus \textbf{tu}is \textbf{sem}per.\

118. Sprevísti omnes discedéntes a ju\textbf{dí}ciis \textbf{tu}is: \ast\  quia injústa cogitáti\textbf{o} e\textbf{ó}rum.\

119. Prævaricántes reputávi omnes pecca\textbf{tó}res \textbf{ter}ræ: \ast\  ídeo diléxi testi\textbf{mó}nia \textbf{tu}a.\

120. Confíge timóre tuo \textbf{car}nes \textbf{me}as: \ast\  a judíciis enim \textbf{tu}is \textbf{tí}mui.\

121. Feci judícium \textbf{et} jus\textbf{tí}tiam: \ast\  non tradas me calumni\textbf{án}ti\textbf{bus} me.\

122. Súscipe servum \textbf{tu}um in \textbf{bo}num: \ast\  non calumniéntur \textbf{me} su\textbf{pér}bi.\

123. Oculi mei defecérunt in salu\textbf{tá}re \textbf{tu}um: \ast\  et in elóquium jus\textbf{tí}tiæ \textbf{tu}æ.\

124. Fac cum servo tuo secúndum miseri\textbf{cór}diam \textbf{tu}am: \ast\  et justificatiónes \textbf{tu}as \textbf{do}ce me.\

125. Servus \textbf{tu}us sum \textbf{e}go: \ast\  da mihi intelléctum, ut sciam testi\textbf{mó}nia \textbf{tu}a.\

126. Tempus faci\textbf{én}di, \textbf{Dó}mine: \ast\  dissipavérunt \textbf{le}gem \textbf{tu}am.\

127. Ideo diléxi man\textbf{dá}ta \textbf{tu}a, \ast\  super aurum \textbf{et} to\textbf{pá}zion.\

128. Proptérea ad ómnia mandáta tua \textbf{di}ri\textbf{gé}bar: \ast\  omnem viam iníquam \textbf{ó}dio \textbf{há}bui.\

129. Mirabília testi\textbf{mó}nia \textbf{tu}a: \ast\  ídeo scrutáta est ea \textbf{á}nima \textbf{me}a.\

130. Declarátio sermónum tu\textbf{ó}rum il\textbf{lú}minat: \ast\  et intel\textbf{léc}tum dat \textbf{pár}vulis.\

131. Os meum apérui, et at\textbf{trá}xi \textbf{spí}ritum: \ast\  quia mandáta tua de\textbf{si}de\textbf{rá}bam.\

132. Aspice in me, et mise\textbf{ré}re \textbf{me}i: \ast\  secúndum judícium diligéntium \textbf{no}men \textbf{tu}um.\

133. Gressus meos dírige secúndum e\textbf{ló}quium \textbf{tu}um: \ast\  et non dominétur mei omnis \textbf{in}jus\textbf{tí}tia.\

134. Rédime me a ca\textbf{lúm}niis \textbf{hó}minum: \ast\  ut custódiam man\textbf{dá}ta \textbf{tu}a.\

135. Fáciem tuam illúmina super \textbf{ser}vum \textbf{tu}um: \ast\  et doce me justificati\textbf{ó}nes \textbf{tu}as.\

136. Exitus aquárum deduxérunt \textbf{ó}culi \textbf{me}i: \ast\  quia non custodiérunt \textbf{le}gem \textbf{tu}am.\

137. \textbf{Jus}tus es, \textbf{Dó}mine: \ast\  et rectum ju\textbf{dí}cium \textbf{tu}um.\

138. Mandásti justítiam testi\textbf{mó}nia \textbf{tu}a: \ast\  et veritátem \textbf{tu}am \textbf{ni}mis.\

139. Tabéscere me fecit \textbf{ze}lus \textbf{me}us: \ast\  quia oblíti sunt verba tua ini\textbf{mí}ci \textbf{me}i.\

140. Ignítum elóquium tuum \textbf{ve}he\textbf{mén}ter: \ast\  et servus tuus di\textbf{lé}xit \textbf{il}lud.\

141. Adolescéntulus sum ego \textbf{et} con\textbf{témp}tus: \ast\  justificatiónes tuas non \textbf{sum} ob\textbf{lí}tus.\

142. Justítia tua, justítia \textbf{in} æ\textbf{tér}num: \ast\  et lex \textbf{tu}a \textbf{vé}ritas.\

143. Tribulátio, et angústia \textbf{in}ve\textbf{né}runt me: \ast\  mandáta tua medi\textbf{tá}tio \textbf{me}a est.\

144. Æquitas testimónia tua \textbf{in} æ\textbf{tér}num: \ast\  intelléctum da \textbf{mi}hi, et \textbf{vi}vam.\

145. Clamávi in toto corde meo, ex\textbf{áu}di me, \textbf{Dó}mine: \ast\  justificatiónes \textbf{tu}as re\textbf{quí}ram.\

146. Clamávi ad te, \textbf{sal}vum \textbf{me} fac: \ast\  ut custódiam man\textbf{dá}ta \textbf{tu}a.\

147. Prævéni in maturitáte, \textbf{et} cla\textbf{má}vi: \ast\  quia in verba tua su\textbf{per}spe\textbf{rá}vi.\

148. Prævenérunt óculi mei ad \textbf{te} di\textbf{lú}culo: \ast\  ut meditárer e\textbf{ló}quia \textbf{tu}a.\

149. Vocem meam audi secúndum misericórdiam \textbf{tu}am, \textbf{Dó}mine: \ast\  et secúndum judícium tuum vi\textbf{ví}fi\textbf{ca} me.\

150. Appropinquavérunt persequéntes me in\textbf{i}qui\textbf{tá}ti: \ast\  a lege autem tua \textbf{lon}ge \textbf{fac}ti sunt.\

151. Prope \textbf{es} tu, \textbf{Dó}mine: \ast\  et omnes viæ \textbf{tu}æ \textbf{vé}ritas.\

152. Inítio cognóvi de testi\textbf{mó}niis \textbf{tu}is: \ast\  quia in ætérnum fun\textbf{dás}ti \textbf{e}a.\

153. Vide humilitátem meam, et \textbf{é}ri\textbf{pe} me: \ast\  quia legem tuam non \textbf{sum} ob\textbf{lí}tus.\

154. Júdica judícium meum, et \textbf{réd}i\textbf{me} me: \ast\  propter elóquium tuum vi\textbf{ví}fi\textbf{ca} me.\

155. Longe a pecca\textbf{tó}ribus \textbf{sa}lus: \ast\  quia justificatiónes tuas non ex\textbf{qui}si\textbf{é}runt.\

156. Misericórdiæ tuæ \textbf{mul}tæ, \textbf{Dó}mine: \ast\  secúndum judícium tuum vi\textbf{ví}fi\textbf{ca} me.\

157. Multi qui persequúntur me, et \textbf{trí}bu\textbf{lant} me: \ast\  a testimóniis tuis non \textbf{de}cli\textbf{ná}vi.\

158. Vidi prævaricántes, et \textbf{ta}be\textbf{scé}bam: \ast\  quia elóquia tua non cus\textbf{to}di\textbf{é}runt.\

159. Vide quóniam mandáta tua di\textbf{lé}xi, \textbf{Dó}mine: \ast\  in misericórdia tua vi\textbf{ví}fi\textbf{ca} me.\

160. Princípium verbórum tu\textbf{ó}rum, \textbf{vé}ritas: \ast\  in ætérnum ómnia judícia jus\textbf{tí}tiæ \textbf{tu}æ.\

161. Príncipes persecúti \textbf{sunt} me \textbf{gra}tis: \ast\  et a verbis tuis formi\textbf{dá}vit cor \textbf{me}um.\

162. Lætábor ego super e\textbf{ló}quia \textbf{tu}a: \ast\  sicut qui invénit \textbf{spó}lia \textbf{mul}ta.\

163. Iniquitátem ódio hábui, et ab\textbf{o}mi\textbf{ná}tus sum: \ast\  legem autem \textbf{tu}am di\textbf{lé}xi.\

164. Sépties in die laudem \textbf{di}xi \textbf{ti}bi, \ast\  super judícia jus\textbf{tí}tiæ \textbf{tu}æ.\

165. Pax multa diligéntibus \textbf{le}gem \textbf{tu}am: \ast\  et non est \textbf{il}lis \textbf{scán}dalum.\

166. Exspectábam salutáre \textbf{tu}um, \textbf{Dó}mine: \ast\  et mandáta \textbf{tu}a di\textbf{lé}xi.\

167. Custodívit ánima mea testi\textbf{mó}nia \textbf{tu}a: \ast\  et diléxit ea \textbf{ve}he\textbf{mén}ter.\

168. Servávi mandáta tua, et testi\textbf{mó}nia \textbf{tu}a: \ast\  quia omnes viæ meæ in con\textbf{spéc}tu \textbf{tu}o.\

169. Appropínquet deprecátio mea in conspéctu \textbf{tu}o, \textbf{Dó}mine: \ast\  juxta elóquium tuum da mihi \textbf{in}tel\textbf{léc}tum.\

170. Intret postulátio mea in con\textbf{spéc}tu \textbf{tu}o: \ast\  secúndum elóquium tuum \textbf{é}ri\textbf{pe} me.\

171. Eructábunt lábia \textbf{me}a \textbf{hym}num, \ast\  cum docúeris me justificati\textbf{ó}nes \textbf{tu}as.\

172. Pronuntiábit lingua mea e\textbf{ló}quium \textbf{tu}um: \ast\  quia ómnia mandáta \textbf{tu}a \textbf{ǽ}quitas.\

173. Fiat manus \textbf{tu}a ut \textbf{sal}vet me: \ast\  quóniam mandáta \textbf{tu}a e\textbf{lé}gi.\

174. Concupívi salutáre \textbf{tu}um, \textbf{Dó}mine: \ast\  et lex tua medi\textbf{tá}tio \textbf{me}a est.\

175. Vivet ánima mea, \textbf{et} lau\textbf{dá}bit te: \ast\  et judícia tua \textbf{ad}ju\textbf{vá}bunt me.\

176. Errávi, sicut \textbf{o}vis, quæ \textbf{pér}iit: \ast\  quære servum tuum, quia mandáta tua non \textbf{sum} ob\textbf{lí}tus.\

177. Glória \textbf{Pa}tri, et \textbf{Fí}lio, \ast\  et Spi\textbf{rí}tui \textbf{Sanc}to.\

178. Sicut erat in princípio, et \textbf{nunc}, et \textbf{sem}per, \ast\  et in sǽcula sæcu\textbf{ló}rum. \textbf{A}men.\

